% !TEX encoding = UTF-8 Unicode
% !TEX program = xelatex
% !BIB program = biber
% !TEX TS-program = xelatex
% !BIB TS-program = biber
%%
%%  本模板方式编译: XeLaTeX + biber
%%
%%  注意: 在改变编译方式前应先删除 *.toc 和 *.aux 文件
%%
\documentclass[12pt,openright]{book}

% 引入NKThesis包
\usepackage[emptydoublepage]{NKThesis}   % 中文
%\usepackage[emptydoublepage,English]{NKThesis} % 英文

% 其它包按需添加
% \usepackage{amsmath}
% \usepackage{cases}
% \usepackage{multirow}
% \usepackage{tocbibind}
\usepackage{ulem}
\usepackage{xeCJKfntef} %CJKunderline
\xeCJKsetup{underline= {format = \color{black}, thickness=0.4pt}} %CJKunderline setting

% 参考文献
\addbibresource{nkthesis.bib}
% 图片文件夹
\graphicspath{{image/}}

\includeonly{
	./tex/abstract,
	./tex/introduction,
	./tex/relatedwork,
	./tex/method,
	./tex/discussion,
	./tex/summary,
	./tex/references,
	./tex/acknowledgements,
	./tex/appendices,
	./tex/resume
}
\begin{document}
%  设置基本信息
%  注意:  逗号`,'是项目分隔符. 如果某一项的值出现逗号, 应放在花括号内, 如 {,}
%
\NKTsetup{
	% 封面设置
	论文题目(中文) = 我是爱南开的,
	论文题目(英文) = I Love Nankai,
	学号           = 19190062,
	姓名       = 周恩来,
	年级          = 1919级,
	学院          = 管理学院,
	系别          = 政治系,
	专业          = 国际政治,
	完成日期      = ,
	指导教师       = 张伯苓教授,
	校外指导教师   = ,
}


\include{./tex/abstract}

\tableofcontents

\include{./tex/introduction}
\include{./tex/relatedwork}
\include{./tex/method}
\include{./tex/discussion}
% !TeX root = ../main.tex
% -*- coding: utf-8 -*-
\chapter{总结展望}

任何问题可在GitHub上发起issue 

\href{https://github.com/kongxiao0532/nku\_bachelor\_thesis/issues/new}{github.com/kongxiao0532/nku\_bachelor\_thesis/issues/new}

\include{./tex/references}
\include{./tex/acknowledgements}
\include{./tex/appendices}

\end{document}
